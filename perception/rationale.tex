\subsection{Rationale}
\label{perception:results:rationale}
Participants were provided a free-text area to optionally provide the reasons behind the choices they made. Below is a selection of these responses, their accompanying posts, and a discussion of what this means relating to the overall results noted above.




\TODO{Coherent}


Participants noted some of the features that they felt made an argument seem credible: either using direct quotes, or deliberately stating that the view was an opinion.

\textit{``Seems like a credible argument by using a quote''}

\textit{``...they're talking about their own opinions so it is credible.''}

Interestingly, participants often remarked that posts they had answered \textit{attempting} to persuade, did not strike them as particularly persuasive:




\TODO{entertaining}

Unsurprisingly, participants broadly felt that posts with foul language were offensive (although this was subjective), or those that directly insulted a person (whether within the discussion, or the topic of it), and often branded them as deliberate trolls. However, in certain cases, participants felt this would actually spur them to reply and engage with the discussion.

\textit{It's mildly offensive, but mostly it's just a bad, totally pointless joke, best ignored.}

\textit{It's just swearing so not particularly offensive.}

\textit{``I don't engage with racial hatred discussions. There's no rational discussion.''}

\textit{``The answer is trolling. Don't feed trolls.''}

\textit{``Essentially a trolling answer.''}

\textit{People are facing execution, and someone posts a dumb joke? It would be pretty funny in other contexts, but this is gross. I'd be more likely to reply just to call them out for being an ass.}


Participants had different opinions on how emotional language would change their behaviour; one explained that they were more likely to reply to posts that didn't seem emotionally charged, whereas another felt the opposite.

\textit{``I liked the non-emotional tone. So I would comfortable replying. But because my emotions are not engaged, I am less inclined to share''}

\textit{``It is emotionally engaging so prompts replying. It is entertaining that promots} [sic] \textit{sharing.''} 

There was also a relatively consistent consensus that participants were more likely to find a post persuasive, or vote for content, if they personally agreed with:

\textit{``The bias is: I am much more likely to share/upvote a comment if I agree with its contents.''}

\textit{``...Since I agree with the comment, I'm likely for me to vote up...''}

\textit{``...I tend to upvote content that I agree with and share content I disagree with...''}

As might be expected, participants were more likely to report posts that did not appear to be entering the discussion in good faith, whether through insults or derailing the topic.

\textit{``Baiting.''}

\textit{``...the respondent isn't likely to engage in polite debate.''}

\textit{``Appears to be spam''}


Several of the rationales given justified the low engagement scores given (an average of \textless2 for replying, sharing, etc.) These were broadly in two camps: either due a general disinterest in the subject at hand, or due to the post being unclear or not credible.

\textit{``I don't really reply to comments on social media, but do often read them. Hence my 'neutral' more/less likely answers to these questions.''}

\textit{``I just don't care about politics.''}

\textit{``...I'm not particularly interested so I wouldn't be likely to engage them.''}

\textit{``Not clear the respondent's intention''}

\textit{``Don't know how credible this information is, so I wouldn't interact with this comment''}

Conversely, some participants explained their enthusiasm for interacting with certain comments particularly \textit{because} of this.

\textit{``This is a stupid argument, so I'm likely to interact with it.''}

Others pointed out they were more likely to interact with posts that appeared to have a central conclusion that could actually be argued for or against.

\textit{``The comment has facts that can be argued for or against - so I'd be more likely to interact with this comment.''}

\textit{``I don't know a lot about the event but I would be more likely to respond to this as there is a clear point that could be discussed...''}