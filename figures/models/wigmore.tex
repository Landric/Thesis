
\makeatletter
\newif\ifpgfshapebaselesstrianglehasinline
\newif\ifpgfshapebaselesstriangleclose
\pgfkeys{/pgf/.cd,
  baseless triangle apex angle/.style={/pgf/isosceles triangle apex angle=#1},
  baseless triangle inline/.is if=pgfshapebaselesstrianglehasinline,
  baseless triangle has base/.is if=pgfshapebaselesstriangleclose
}

\pgfdeclareshape{baseless triangle}{
  % Copy some stuff from the isosecles triangle
  \inheritsavedanchors[from={isosceles triangle}]
  \inheritanchor[from={isosceles triangle}]{center}
  \inheritanchor[from={isosceles triangle}]{north}
  \inheritanchor[from={isosceles triangle}]{south}
  \inheritanchor[from={isosceles triangle}]{east}
  \inheritanchor[from={isosceles triangle}]{west}
  \inheritanchorborder[from={isosceles triangle}]
  \backgroundpath{%
    % The isoceles triangle defines lots of parameters
    % in the \trianglepoints macro.
        \trianglepoints%
        {%
            \pgftransformshift{\centerpoint}%
            \pgftransformrotate{\rotate}%
            % This bit is a bit of a kludge to ensure the inline
            % is at the top of the figure.
            \pgftransformyscale{cos(\rotate)}%
            \pgfpathmoveto{\lowerleft}%
            \pgfpathlineto{\apex}%
            \pgfpathlineto{\lowerleft\pgf@y=-\pgf@y}%
            % Close the base?
            \ifpgfshapebaselesstriangleclose%
              \pgfpathclose%
            \fi%
            % Draw the inline?
            \ifpgfshapebaselesstrianglehasinline
              \pgfpointdiff{\lowerleft}%
                 {\pgfpointlineattime{0.125}{\lowerleft}{\lowerleft\pgf@y=-\pgf@y}}%
                \pgfgetlastxy{\x}{\y}%
                \pgfmathveclen{\x}{\y}%
                \let\inlineshift=\pgfmathresult%
            % Calculate where the inline hits the sloped line of the triangle.
            \pgfmathparse{\inlineshift/2/sin(\pgfkeysvalueof{/pgf/isosceles triangle apex angle}/2)}%
            \let\inlineendshift=\pgfmathresult
            \pgfpathmoveto{\pgfpointadd{\pgfpoint{0pt}{-\inlineshift}}{\lowerleft}}%
            \pgfpathlineto{\pgfpointlineatdistance{\inlineendshift}{\apex}{\lowerleft}\pgf@y=-\pgf@y}%
        \fi%
    }
    }
}

\begin{center}
\begin{tikzpicture}[node distance=1.6cm,>=stealth',bend angle=45]

	\tikzstyle{circ}=[
		circle,
		thick,
		draw=black,
		fill=white,
		minimum size=6mm
	]

	\tikzstyle{rect}=[
		rectangle,
		thick,
		draw=black,
  		fill=white,
  		minimum size=6mm
  	]
  			  
	\tikzset{
		support/.style={
			regular polygon,
			regular polygon sides=3,
			shape border rotate=90,
			thick,
        	draw=black,
	        fill=white,
        	minimum height=6mm,
        	node distance=0.6cm
    	}
    }
    
    \tikzset{	
		refute/.style={
	    	draw,
	    	thick,
	    	baseless triangle,
 	   		baseless triangle apex angle=60,
 	   		shape border rotate=180,
 	   		baseless triangle inline=false,
 	   		baseless triangle has base=false,
 	   		node distance=0.6cm
  		}
  	}



	\node [circ,tokens=1] (1) [label=above right:$_1$] {};
	
	\node [circ,tokens=2] (2) [right of=1,label=above right:$_2$,label=below right:$\infty$] {};
	\node [support] 	  (2a)[left of=2] {};	
	
	\node [circ,tokens=1] (3) [below left of=1,label=above right:$_3$] {};	
	\node [circ,tokens=2] (4) [below of=3,label=above right:$_4$,label=below right:\footnotesize{\P}] {};

	\node [circ,tokens=1] (5) [below right of=1,label=above right:$_5$] {};
	


	\node [rect,tokens=1] (6) [below of=5,label=above right:$_6$] {};	

	\node [circ] (7)  [right of=6,label=above right:$_7$] {$\circ$};
	\node [refute] (7a)  [left of=7] {};

	\draw[post] (3)-|(1)  ;
	\draw[post] (5)-|(1)  ;
	\draw[post] (4)--(3)  ;
	\draw[post] (6)--(5)  ;
	\draw[post] (2a)--(1)  ;
	\draw[post] (7a)--(6)  ;
	
 
      
\end{tikzpicture}
\end{center}

\begin{tabular}{l l}
\begin{minipage}{0.5\textwidth}
\begin{enumerate}
	\item[$^1$] \footnotesize{Alice is a British citizen}
	\item[$^2$] \footnotesize{Alice has a British passport}
	\item[$^3$] \footnotesize{A person born in a British territory will be a British citizen}
	\item[$^4$] \footnotesize{British Overseas Territories Act 2002}
\end{enumerate}
\end{minipage}

\begin{minipage}{0.5\textwidth}
\begin{enumerate}
	\item[$^5$] \footnotesize{Alice was born in Bermuda}
	\item[$^6$] \footnotesize{Alice's parents testify that she was born in Bermuda}
	\item[$^7$] \footnotesize{Alice's parents' testimony could be biased in her favour}
\end{enumerate}
\end{minipage}
\end{tabular}